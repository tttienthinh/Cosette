\documentclass[10pt]{article}
% TITLE
\title{Cosette Token}

% AUTHORS
\author{TRAN-THUONG Tien-Thinh}


% DATE
\date{2021/05}

\begin{document}
\maketitle

% ABSTRACT
\begin{abstract}
Nous utilisons tous les réseaux sociaux. Certain postent régulièrement et ont beaucoup d'ami qui les suivent. Instagram est l'un de ses réseaux dont la majorité ont entre 18 et 34 ans. Un publique jeune, souvent interéressé par la cryptomonnaie et qui souhaite souvent avoir un complément de revenu à partir de leur activité sur Instagram. Cependant les agences de marketing sur les réseaux sociaux ne cherchent souvent que des comptes avec un minimum de suiveurs pour leur proposer des posts rémunérer.
\end{abstract}

% MAIN BODY
\section{Introduction}
Nous pensons qu'il est plus avantageux pour les entrprises à la recherche d'un poste viral, que celui-ci soit poster par plusieurs "petits" comptes plutôt qu'un "gros" compte d'influenceur. En effet, les plus petits comptes sont sont en majorité suivis par des amis et des proches, ce qui en fait une audience plus à l'écoute. Notre service permet donc aux entreprises de lancer une campagne de publicité sur un certain temps, et à la date butoire, les comptes ayant participé en postant l'image recevront une quantité de Token proportionnellement à leur taux d'engagement sur le post.

\section{Methods}
Nous utiliserons pour cela le langage solidity.


\section*{Acknowledgements}
Merci à celles et ceux qui ont participé à ce projet.

% REFERENCE LIST
\begin{thebibliography}{99}

\bibitem{https://ethereum.org/en/whitepaper/}
Ethereum Whitepaper.

\end{thebibliography}

%%% PRACE GENERIC LAYOUT; DO NOT CHANGE %%%
\end{document}
%%% END OF PRACE GENERIC LAYOUT %%%